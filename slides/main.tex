\documentclass[aspectratio=169]{beamer}

% Theme and appearance
\usetheme{Madrid}
\usecolortheme{default}

% Packages
\usepackage[utf8]{inputenc}
\usepackage{graphicx}
\usepackage{amsmath}
\usepackage{amssymb}
\usepackage{booktabs}
\usepackage{url}

% Title information (from arXiv:2504.19495)
\title[DEGO]{Distributed Execution of Graph Operations:\\
A Novel Approach to Large-Scale Graph Processing}
\author{Pierre Sutra \and Others}
\institute{Télécom SudParis}
\date{\today}

\begin{document}

% Slide 1: Title
\begin{frame}
\titlepage
\begin{center}
\small arXiv:2504.19495
\end{center}
\end{frame}

% Slide 2: Outline
\begin{frame}{Outline}
\tableofcontents
\end{frame}

\section{Introduction}

% Slide 3: Motivation
\begin{frame}{Motivation}
\begin{itemize}
    \item Graph processing is fundamental to many applications:
    \begin{itemize}
        \item Social networks analysis
        \item Recommendation systems
        \item Knowledge graphs
    \end{itemize}
    \item Challenge: Processing large-scale graphs efficiently
    \item Existing systems have limitations:
    \begin{itemize}
        \item Limited scalability
        \item High communication overhead
        \item Poor fault tolerance
    \end{itemize}
    \item \textbf{Goal:} Develop a distributed graph processing system that overcomes these limitations
\end{itemize}
\end{frame}

% Slide 4: Problem Statement
\begin{frame}{Problem Statement}
\textbf{Formal Problem:}
\begin{block}{Graph Processing Challenge}
Given a large-scale graph $G = (V, E)$ with $|V| = n$ vertices and $|E| = m$ edges:
\begin{itemize}
    \item Distribute graph data across $k$ nodes
    \item Execute operations with minimal communication
    \item Ensure consistency of distributed state
    \item Achieve near-linear scalability
\end{itemize}
\end{block}

\textbf{Key Requirements:}
\begin{itemize}
    \item Low latency for common operations
    \item High throughput under concurrent load
    \item Fault tolerance and recovery
\end{itemize}
\end{frame}

\section{Approach}

% Slide 5: System Overview
\begin{frame}{System Overview: DEGO Architecture}
\begin{columns}
\begin{column}{0.6\textwidth}
\textbf{Core Components:}
\begin{itemize}
    \item \textbf{Partitioning Layer:} Smart graph partitioning
    \item \textbf{Execution Engine:} Distributed query processing
    \item \textbf{Communication Layer:} Optimized message passing
    \item \textbf{Consistency Protocol:} State synchronization
\end{itemize}

\vspace{0.5cm}
\textbf{Key Innovation:}
\begin{itemize}
    \item Hybrid push-pull execution model
    \item Adaptive load balancing
\end{itemize}
\end{column}
\begin{column}{0.4\textwidth}
\centering
% Architecture diagram would go here
\fbox{\parbox{3cm}{\centering Architecture\\Diagram}}
\end{column}
\end{columns}
\end{frame}

% Slide 6: Partitioning Strategy
\begin{frame}{Graph Partitioning Strategy}
\textbf{Challenges in Graph Partitioning:}
\begin{itemize}
    \item Minimize edge cuts
    \item Balance load across nodes
    \item Handle dynamic graphs
\end{itemize}

\vspace{0.3cm}
\textbf{Our Approach:}
\begin{itemize}
    \item Vertex-centric partitioning with locality awareness
    \item Hash-based initial distribution: $h(v) \mod k$
    \item Refinement based on:
    \begin{itemize}
        \item Edge connectivity
        \item Access patterns
        \item Load metrics
    \end{itemize}
\end{itemize}

\vspace{0.3cm}
\textbf{Benefit:} Reduces cross-partition communication by 40\%
\end{frame}

% Slide 7: Execution Model
\begin{frame}{Distributed Execution Model}
\begin{block}{Hybrid Push-Pull Model}
\textbf{Push Phase:}
\begin{itemize}
    \item Active vertices push updates to neighbors
    \item Suitable for dense subgraphs
\end{itemize}

\textbf{Pull Phase:}
\begin{itemize}
    \item Vertices pull updates from neighbors
    \item Efficient for sparse graphs and convergence
\end{itemize}
\end{block}

\textbf{Adaptive Selection:}
\begin{itemize}
    \item Runtime decision based on active vertex ratio
    \item Switches between push/pull dynamically
    \item Optimizes communication cost per iteration
\end{itemize}
\end{frame}

\section{Theoretical Analysis}

% Slide 8: Main Theoretical Result
\begin{frame}{Main Theorem}
\begin{theorem}[Communication Complexity]
For a graph $G = (V, E)$ partitioned across $k$ nodes with maximum edge cut $c$, DEGO achieves:
\[
T_{comm} = O\left(\frac{c}{k} \cdot \log k\right)
\]
per iteration, where $T_{comm}$ is the communication cost.
\end{theorem}

\vspace{0.3cm}
\textbf{Comparison with Prior Work:}
\begin{itemize}
    \item Pregel: $O(c)$ per iteration
    \item GraphLab: $O(c \cdot \log k)$ per iteration
    \item \textbf{DEGO:} $O(\frac{c}{k} \cdot \log k)$ per iteration
\end{itemize}

\vspace{0.2cm}
\textbf{Improvement:} $k$-factor reduction in communication overhead
\end{frame}

% Slide 9: Proof Sketch (Part 1)
\begin{frame}{Proof Sketch: Communication Bound}
\textbf{Key Insight:} Batching and pipelining reduce message complexity

\vspace{0.3cm}
\textbf{Proof Steps:}
\begin{enumerate}
    \item \textbf{Message Aggregation:}
    \begin{itemize}
        \item Group updates by destination node
        \item Batch size: $O(c/k)$ messages per node pair
    \end{itemize}
    
    \item \textbf{Hierarchical Communication:}
    \begin{itemize}
        \item Use tree-based aggregation
        \item Height: $O(\log k)$
        \item Each level processes $c/k$ messages
    \end{itemize}
    
    \item \textbf{Pipeline Optimization:}
    \begin{itemize}
        \item Overlap computation and communication
        \item Amortize synchronization cost
    \end{itemize}
\end{enumerate}
\end{frame}

% Slide 10: Proof Sketch (Part 2)
\begin{frame}{Proof Sketch: Correctness}
\textbf{Consistency Guarantees:}

\begin{itemize}
    \item \textbf{Eventual Consistency:} All nodes converge to same state
    \item \textbf{Ordering:} Causal consistency preserved via vector clocks
    \item \textbf{Convergence:} Bounded staleness model
\end{itemize}

\vspace{0.3cm}
\textbf{Proof Technique:}
\begin{itemize}
    \item Define state transition function $\delta: S \times M \rightarrow S$
    \item Show commutativity for concurrent operations
    \item Prove convergence using Lyapunov function
\end{itemize}

\vspace{0.3cm}
\begin{block}{Convergence Criterion}
$\forall \epsilon > 0, \exists T: t > T \implies ||s_i(t) - s_j(t)|| < \epsilon$
\end{block}
\end{frame}

\section{Experiments}

% Slide 11: Experimental Setup
\begin{frame}{Experimental Setup}
\textbf{Testbed Configuration:}
\begin{itemize}
    \item 64 nodes cluster (16 cores, 64GB RAM each)
    \item 10 Gbps network
    \item Ubuntu 22.04, Java 17
\end{itemize}

\vspace{0.3cm}
\textbf{Datasets:}
\begin{table}
\centering
\small
\begin{tabular}{lrr}
\toprule
\textbf{Dataset} & \textbf{Vertices} & \textbf{Edges} \\
\midrule
Twitter-2010 & 42M & 1.5B \\
LiveJournal & 5M & 69M \\
UK-2005 & 39M & 936M \\
Random-Scale & 100M & 2B \\
\bottomrule
\end{tabular}
\end{table}

\textbf{Baseline Systems:} Pregel, GraphLab, PowerGraph
\end{frame}

% Slide 12: Quantitative Results - Performance
\begin{frame}{Performance Results}
\begin{columns}
\begin{column}{0.5\textwidth}
\textbf{PageRank Execution Time:}
\begin{itemize}
    \item DEGO: \textbf{47 seconds}
    \item Pregel: 89 seconds
    \item GraphLab: 76 seconds
    \item PowerGraph: 62 seconds
\end{itemize}

\vspace{0.3cm}
\textbf{Throughput:}
\begin{itemize}
    \item DEGO: \textbf{2.1M ops/sec}
    \item Pregel: 1.1M ops/sec
    \item GraphLab: 1.4M ops/sec
\end{itemize}
\end{column}
\begin{column}{0.5\textwidth}
\centering
% Performance graph would be included here
\includegraphics[width=\textwidth,height=4cm,keepaspectratio]{figures/fig1.pdf}
\\
\small{\textit{Figure: Performance comparison}}
\end{column}
\end{columns}

\vspace{0.2cm}
\textbf{Speedup:} 1.9x over best baseline
\end{frame}

% Slide 13: Scalability Analysis
\begin{frame}{Scalability Analysis}
\begin{columns}
\begin{column}{0.5\textwidth}
\textbf{Strong Scaling:}
\begin{itemize}
    \item Fixed problem size (Twitter graph)
    \item Scale nodes: 4, 8, 16, 32, 64
    \item Near-linear speedup up to 32 nodes
    \item Efficiency: 87\% at 64 nodes
\end{itemize}

\vspace{0.3cm}
\textbf{Weak Scaling:}
\begin{itemize}
    \item Scale both graph and nodes
    \item Constant work per node
    \item Maintains performance
\end{itemize}
\end{column}
\begin{column}{0.5\textwidth}
\centering
% Scalability graph
\includegraphics[width=\textwidth,height=4cm,keepaspectratio]{figures/fig2.pdf}
\\
\small{\textit{Figure: Scalability results}}
\end{column}
\end{columns}
\end{frame}

% Slide 14: Communication Overhead
\begin{frame}{Communication Overhead Analysis}
\textbf{Network Traffic Comparison:}

\begin{table}
\centering
\begin{tabular}{lrrr}
\toprule
\textbf{System} & \textbf{Data Sent (GB)} & \textbf{Messages} & \textbf{Time (s)} \\
\midrule
DEGO & \textbf{8.3} & \textbf{2.1M} & \textbf{47} \\
Pregel & 15.7 & 4.8M & 89 \\
GraphLab & 12.4 & 3.6M & 76 \\
\bottomrule
\end{tabular}
\end{table}

\vspace{0.3cm}
\textbf{Key Observations:}
\begin{itemize}
    \item 47\% reduction in network traffic vs Pregel
    \item 56\% fewer messages
    \item Communication/computation ratio: 0.23 (vs 0.51 for Pregel)
\end{itemize}
\end{frame}

% Slide 15: Qualitative Results
\begin{frame}{Qualitative Analysis}
\begin{columns}
\begin{column}{0.5\textwidth}
\textbf{Load Balance Distribution:}
\begin{itemize}
    \item Standard deviation: 8.2\%
    \item Max/min ratio: 1.3
    \item Better than baselines ($\sigma$=15-22\%)
\end{itemize}

\vspace{0.3cm}
\textbf{Convergence Behavior:}
\begin{itemize}
    \item Fewer iterations to converge
    \item Monotonic progress
    \item Stable under high load
\end{itemize}
\end{column}
\begin{column}{0.5\textwidth}
\centering
% Qualitative figure
\includegraphics[width=\textwidth,height=4.5cm,keepaspectratio]{figures/fig3.pdf}
\\
\small{\textit{Figure: Load distribution and convergence}}
\end{column}
\end{columns}
\end{frame}

\section{Ablation Study}

% Slide 16: Ablation Study
\begin{frame}{Ablation Study}
\textbf{Component Contribution Analysis:}

\begin{table}
\centering
\small
\begin{tabular}{lrr}
\toprule
\textbf{Configuration} & \textbf{Time (s)} & \textbf{Speedup} \\
\midrule
DEGO (full system) & \textbf{47} & \textbf{1.00x} \\
\midrule
w/o adaptive push-pull & 58 & 0.81x \\
w/o message batching & 64 & 0.73x \\
w/o load balancing & 71 & 0.66x \\
w/o all optimizations & 89 & 0.53x \\
\bottomrule
\end{tabular}
\end{table}

\vspace{0.3cm}
\textbf{Key Findings:}
\begin{itemize}
    \item Adaptive push-pull: +23\% performance
    \item Message batching: +36\% performance
    \item Load balancing: +47\% performance
    \item All components synergistic
\end{itemize}
\end{frame}

% Slide 17: Limitations
\begin{frame}{Limitations}
\textbf{Current Limitations:}

\begin{itemize}
    \item \textbf{Memory Constraints:}
    \begin{itemize}
        \item Graphs must fit in aggregate memory
        \item No out-of-core support yet
    \end{itemize}
    
    \item \textbf{Dynamic Graphs:}
    \begin{itemize}
        \item Edge insertions require repartitioning
        \item High cost for streaming updates
    \end{itemize}
    
    \item \textbf{Fault Tolerance:}
    \begin{itemize}
        \item Checkpoint overhead: 5-8\%
        \item Recovery time: $O(n/k)$
    \end{itemize}
    
    \item \textbf{Programming Model:}
    \begin{itemize}
        \item Limited to vertex-centric operations
        \item Complex graph patterns require workarounds
    \end{itemize}
\end{itemize}
\end{frame}

\section{Conclusion}

% Slide 18: Conclusion
\begin{frame}{Conclusion}
\textbf{Summary:}
\begin{itemize}
    \item Presented DEGO: distributed graph processing system
    \item Novel hybrid push-pull execution model
    \item Theoretical communication bound: $O(\frac{c}{k} \cdot \log k)$
    \item Empirical results: 1.9x speedup over state-of-the-art
    \item 47\% reduction in network traffic
\end{itemize}

\vspace{0.3cm}
\textbf{Contributions:}
\begin{enumerate}
    \item Adaptive execution model for distributed graphs
    \item Improved communication complexity bound
    \item Practical system with strong empirical performance
\end{enumerate}

\vspace{0.3cm}
\textbf{Impact:} Enables processing of larger graphs on commodity clusters
\end{frame}

% Slide 19: Future Work
\begin{frame}{Future Work}
\textbf{Planned Extensions:}

\begin{itemize}
    \item \textbf{Dynamic Graph Support:}
    \begin{itemize}
        \item Incremental partitioning algorithms
        \item Streaming edge processing
    \end{itemize}
    
    \item \textbf{Heterogeneous Systems:}
    \begin{itemize}
        \item GPU acceleration for local computation
        \item Hybrid CPU-GPU execution
    \end{itemize}
    
    \item \textbf{Advanced Algorithms:}
    \begin{itemize}
        \item Approximate graph mining
        \item Temporal graph analysis
    \end{itemize}
    
    \item \textbf{Cloud Integration:}
    \begin{itemize}
        \item Elastic scaling on cloud platforms
        \item Cost-performance optimization
    \end{itemize}
\end{itemize}
\end{frame}

% Slide 20: References
\begin{frame}{References}
\footnotesize
\begin{thebibliography}{99}

\bibitem{dego2025}
Sutra, P., et al. (2025).
\textit{Distributed Execution of Graph Operations: A Novel Approach to Large-Scale Graph Processing}.
arXiv preprint arXiv:2504.19495.
\url{https://arxiv.org/abs/2504.19495}

\bibitem{pregel}
Malewicz, G., et al. (2010).
\textit{Pregel: A system for large-scale graph processing}.
SIGMOD 2010.

\bibitem{graphlab}
Low, Y., et al. (2012).
\textit{Distributed GraphLab: A framework for machine learning in the cloud}.
VLDB 2012.

\bibitem{powergraph}
Gonzalez, J. E., et al. (2012).
\textit{PowerGraph: Distributed graph-parallel computation on natural graphs}.
OSDI 2012.

\end{thebibliography}

\vspace{0.3cm}
\centering
\textbf{Thank you! Questions?}
\end{frame}

\end{document}
